% Options for packages loaded elsewhere
\PassOptionsToPackage{unicode}{hyperref}
\PassOptionsToPackage{hyphens}{url}
%
\documentclass[
]{article}
\author{}
\date{}

\usepackage{amsmath,amssymb}
\usepackage{lmodern}
\usepackage{iftex}
\ifPDFTeX
  \usepackage[T1]{fontenc}
  \usepackage[utf8]{inputenc}
  \usepackage{textcomp} % provide euro and other symbols
\else % if luatex or xetex
  \usepackage{unicode-math}
  \defaultfontfeatures{Scale=MatchLowercase}
  \defaultfontfeatures[\rmfamily]{Ligatures=TeX,Scale=1}
\fi
% Use upquote if available, for straight quotes in verbatim environments
\IfFileExists{upquote.sty}{\usepackage{upquote}}{}
\IfFileExists{microtype.sty}{% use microtype if available
  \usepackage[]{microtype}
  \UseMicrotypeSet[protrusion]{basicmath} % disable protrusion for tt fonts
}{}
\makeatletter
\@ifundefined{KOMAClassName}{% if non-KOMA class
  \IfFileExists{parskip.sty}{%
    \usepackage{parskip}
  }{% else
    \setlength{\parindent}{0pt}
    \setlength{\parskip}{6pt plus 2pt minus 1pt}}
}{% if KOMA class
  \KOMAoptions{parskip=half}}
\makeatother
\usepackage{xcolor}
\IfFileExists{xurl.sty}{\usepackage{xurl}}{} % add URL line breaks if available
\IfFileExists{bookmark.sty}{\usepackage{bookmark}}{\usepackage{hyperref}}
\hypersetup{
  hidelinks,
  pdfcreator={LaTeX via pandoc}}
\urlstyle{same} % disable monospaced font for URLs
\setlength{\emergencystretch}{3em} % prevent overfull lines
\providecommand{\tightlist}{%
  \setlength{\itemsep}{0pt}\setlength{\parskip}{0pt}}
\setcounter{secnumdepth}{-\maxdimen} % remove section numbering
\ifLuaTeX
  \usepackage{selnolig}  % disable illegal ligatures
\fi

\begin{document}

{
\setcounter{tocdepth}{3}
\tableofcontents
}
\hypertarget{equazioni-di-secondo-grado}{%
\section{Equazioni di secondo grado}\label{equazioni-di-secondo-grado}}

\hypertarget{definizione}{%
\subsection{Definizione}\label{definizione}}

Un equazione è un ugualianza tra due espressioni numerico-letterali, in
cui il primo membro (l'espressione a sinistra) è perfettamente uguale al
secondo membro (l'espressione a destra)

\hypertarget{metodo-risolutivo}{%
\subsection{Metodo risolutivo}\label{metodo-risolutivo}}

\begin{quote}
\[ax^2+bx+c=0 \]
\end{quote}

\begin{enumerate}
\def\labelenumi{\arabic{enumi}.}
\item
  Calcolo del \textbf{Delta discriminante} (\(\Delta\))
  \(\Delta = b^2-4ac\)
\item
  Calcolo della soluzione \(x_{1/2} = \frac{-b\pm\sqrt{\Delta}}{2a}\)
\end{enumerate}

\hypertarget{eccezioni}{%
\subsection{Eccezioni}\label{eccezioni}}

\begin{enumerate}
\def\labelenumi{\arabic{enumi}.}
\tightlist
\item
  Delta = 0 -\textgreater{} le soluzioni sono \emph{coincidenti} o la
  soluzione e' doppia
\item
  Delta negativo -\textgreater{} l'equazione è \emph{impossibile}
\end{enumerate}

\hypertarget{esempio}{%
\subsection{Esempio}\label{esempio}}

\hypertarget{x23x-10}{%
\paragraph{\texorpdfstring{\(4x^2+3x-1=0\)}{4x\^{}2+3x-1=0}}\label{x23x-10}}

\begin{quote}
{[}!NOTE{]} Soluzione \(\Delta = (+3)^2 - 4(+4)(-1) = +9+16 = 25\)
\(x_{1/2} = \frac{-(+3)\pm\sqrt{25}}{2(4)} = \frac{-3\pm5}{8}\)
\(x_1 = \frac{-3+5}{8} = \frac28 = \frac14\)
\(x_2 = \frac{-3-5}{8} = \frac{-8}{8} = -1\)
\end{quote}

\hypertarget{casi-particolari}{%
\subsection{Casi particolari}\label{casi-particolari}}

\hypertarget{equazione-pura-coefficiente-di-b-0}{%
\subsubsection{\texorpdfstring{Equazione pura (Coefficiente di
\(b =0\))}{Equazione pura (Coefficiente di b =0)}}\label{equazione-pura-coefficiente-di-b-0}}

\hypertarget{x2-40}{%
\subparagraph{\texorpdfstring{\(x^2-4=0\)}{x\^{}2-4=0}}\label{x2-40}}

\hypertarget{soluzione-metodo-tradizionale}{%
\subparagraph{Soluzione (metodo
tradizionale)}\label{soluzione-metodo-tradizionale}}

\(\Delta = -4ac = 16\) \(x_{1/2}=\pm\frac{4}{2}\) \(x_1 = -2\)
\(x_2 = 2\)

\begin{quote}
{[}!NOTE{]} Soluzione (metodo abbreviato) \(x^2 = 4\)
\(\sqrt{x^2} = \pm\sqrt4\) \(x_{1/2} = \pm2\)
\end{quote}

\hypertarget{equazione-spuria-coefficente-di-c0}{%
\subsubsection{\texorpdfstring{Equazione spuria (coefficente di
\(c=0\))}{Equazione spuria (coefficente di c=0)}}\label{equazione-spuria-coefficente-di-c0}}

\hypertarget{x2-7x0}{%
\subparagraph{\texorpdfstring{\(x^2-7x=0\)}{x\^{}2-7x=0}}\label{x2-7x0}}

\begin{quote}
{[}!NOTE{]} Soluzione (metodo tradizionale) \(\Delta = 49\)
\(x_{1/2}=\frac{7\pm7}{2}\) \(x_1 = 0\) \(x_2 = 7\)
\end{quote}

\begin{quote}
{[}!NOTE{]} Soluzione (metodo abbreviato) \(x(x-7) = 0\)

\(x_1 = 0\) \(x_2 = 7\)
\end{quote}

\end{document}
